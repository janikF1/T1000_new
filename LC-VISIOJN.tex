% !TEX root = 4268671_Janik_Frick_T1000.tex
\section{LC-VISION}
\glqq LC-VISION\grqq\space ist eine Software von \ac{Blum} für die Visualisierung und Auswertung von Messdaten der \glqq DIGILOG-Lasermesssysteme\grqq.
Bei der Entwicklung der Software treten Unstimmigkeiten auf, die die Funktionalität der Software nicht beeinträchtigen, aber dennoch behoben werden sollten.
Während der Behebung von Unstimmigkeiten kann man von der Erfahrung anderer Entwickler profitieren. Da die Lösung in den bestehenden Code integriert wird, kommt man mit den Techniken anderer Entwickler in Kontakt. Die eigene Arbeit kann reflektiert werden\cite{34908}, wenn darüber gesprochen wird. Außerdem fördert die aktive Mitarbeit nicht nur Kenntnisse über die Theorie der Softwareentwicklung, sondern auch über die praktischen Aspekte\cite{34908}.\\
Ein weiterer Vorteil ist es, dass der Lernprozess in einer relevanten Umgebung stattfindet\cite{34908}.
Hat eine Aufgabe Relevanz, kann dies einen positiven Einfluss auf die Motivation haben. \\
Bevor mit der Behebung von Unstimmigkeiten angefangen werden konnte, sollte man den Code und die Benutzeroberfläche kennenzulernen. \\
Um die Benutzeroberfläche kennenzulernen, kann eine Kombination aus statischer und dynamischer Analyse verwendet werden. Bei der statischen Analyse wird nur der Code betrachtet. Um hieraus qualitative Informationen zu gewinnen, wird ein großer Teil des Codes benötigt, da nur so Zusammenhänge und Abläufe erkannt werden können\cite{mock2003dynamic}. Der Vorteil liegt darin, dass dieses Verfahren auf jedem System durchgeführt werden kann, das einen Texteditor bietet. \\
In der dynamischen Analyse werden Informationen während der Laufzeit der Anwendung gesammelt. Die dynamische Analyse bringt häufig die besseren Ergebnisse, da viele Abhängigkeiten, zum Beispiel \glqq dynamic link libraries\grqq, erst zur Laufzeit eingebunden werden.
Mit diesen Analyseverfahren kann der Code optimiert werden und das Verständnis des Codes verbessert werden\cite{mock2003dynamic}.\\
Für die Einarbeitung in die Software \glqq LC-VISION\grqq\space liegt der Fokus auf der Verbesserung für das Verständnis der Abläufe. Hierzu kommt eine Kombination von statischer und dynamischer Analyse zum Einsatz. Um die obersten Schichten der Software ausfindig zu machen, wurde die dynamische Analyse genutzt. In der gestarteten Anwendung wurde nach Elementen gesucht, die im Code leicht wiederzufinden sind. Solche Elemente können Texte, Bilder oder Menüabläufe sein. Werden Muster aus der \ac{GUI} im Code gefunden, kann man von dort aus Abläufen in tiefere Schichten der Software folgen. Dieser Schritt ist Teil der zuvor erwähnten statischen Analyse. Treten hierbei Probleme oder Fragen auf, kann es hilfreich sein, mit einen Entwickler der Software zu sprechen, ob dieser Unterstützung bieten kann.\\
Das erworbene Wissen kann durch die Behebung von Unstimmigkeiten, wie der Anpassung von Texten und Übersetzungen, sowie inkonsistenter Datenanzeige in Sonderfällen, gefestigt werden.\\
Danach kann an Erweiterungen oder der Umstrukturierung der Software gearbeitet werden. In diesem Fall bestehen die Aufgaben aus Anpassungen und Erweiterungen der Benutzeroberfläche. Ist die Signalstärke des Lasers zu schwach, soll ein Handlungsvorschlag angezeigt werden, mit dem die Signalstärke optimiert werden kann. Eine zweite Anpassung ist eine Fehlerbehebung im Bereich der Lizenzanzeige. Informationen zu selten genutzten Lizenzen sollten nur angezeigt werden, wenn diese gekauft sind. 