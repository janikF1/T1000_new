\documentclass[12pt,a4paper]{article}
\usepackage[utf8]{inputenc}
\usepackage[german]{babel}
\usepackage[T1]{fontenc}
\usepackage{amsmath}
\usepackage{amsfonts}
\usepackage{amssymb}
\usepackage{graphicx}
\author{Janik Frick}
\title{Konzipierung und Erstellung eines Parametertabellenkonfigurators und Praxisnahe Mitarbeit im releasten Produkt LC-VISION}
\usepackage[left=2.5cm,right=2.5cm,top=2.5cm,bottom=2.5cm]{geometry}
\begin{document}
\maketitle
\newpage
\section{Einleitung}
Die Blum-Novotest GmbH(Blum) ist im Bereich Maschinenbau für Mess- und Prüftechnik tätig. Am Standort Grünkraut (Baden-Württemberg) ist der Hauptsitz der Firma. Dort befinden sich Entwicklung und Fertigung für den Bereich Messtechnik. Am zweiten Standort in Willich (Nordrhein-Westfalen) sind die Entwicklung und Fertigung im Bereich Prüftechnik untergebracht.\\
Produkte von Blum werden in vielen anspruchsvollen Industrien im Bereich der Qualitätssicherung eingesetzt. Die Kunden kommen unter anderem aus der Automobil-, Luftfahrt- und der Werkzeugmaschinenindustrie. \\
An die Produkte dieser Industrien werden höchste Ansprüche in Sachen Qualität gestellt. Daraus resultieren auch für die Produkte der Firma Blum höchste Ansprüche. \\
Um diese Ansprüche erfüllen zu können, werden sowohl bestehende Produkte laufend optimiert und weiterentwickelt, als auch neue Produkte entwickelt. \\
Die Arbeit wird im Umfeld von Blum am Standort Grünkraut verfasst. \\
Ziel der Arbeit ist es die vielfältigen Aspekte der Softwareentwicklung kennenzulernen und den Einstieg zu schaffen.\\
Um dieses Ziel zu erreichen, besteht der praktische Teil aus zwei Projekten.\\
Mit der Entwicklung eines "Parametertabellenkonfigurators" sollen die verschiedenen Aufgaben der Softwareentwicklung veranschaulicht werden.\\
Durch die Mitarbeit am schon bestehenden Produkt "LC-Vision" wird die Produktpflege und Weiterentwicklung thematisiert. 
Außerdem wird dabei das Einarbeiten in unbekannten Code relevant, was ein wichtiger Bestandteil der Produktpflege ist.
\newpage
\section{Parametertabellenkonfigurator}
\subsection{Problemstellung}
Bei bestehenden Programmen für numerische Steuerungen müssen während der Inbetriebnahme Parametertabellen in einem beliebigen Texteditor angepasst werden. Aufgrund der Anzahl, wie auch der Einstellmöglichkeiten der Parameter, wird hierfür eine separate Installationsanleitung benötigt. Der Prozess der Parametrierung ist daher Fehleranfällig und von der Erfahrung des Inbetriebnehmers abhängig. \\
Um die Fehleranfälligkeit dieses Prozesses zu minimieren, soll ein \glqq Parametertabellenkonfigurator\grqq\space entwickelt werden.
\subsection{Anforderungsanalyse}

\end{document}