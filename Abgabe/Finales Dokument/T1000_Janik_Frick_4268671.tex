\documentclass[10pt,a4paper]{article}
\usepackage[utf8]{inputenc}
\usepackage{amsmath}
\usepackage{amsfonts}
\usepackage{amssymb}
\author{Janik Frick}
\title{T1000}
\begin{document}
%TODO Einstellungen machen
%TODO Deckblatt entwerfen
%TODO Eigenarbeitserklärung
%TODO Abkürzungsverzeichnis
%TODO Logos auf jede Seite

\tableofcontents
\newpage
\section{Einleitung}
Die Blum-Novotest GmbH ist im Bereich der Mess- und Prüftechnik tätig. Die Produkte kommen in vielen anspruchsvollen Industrien zum Einsatz, zum Beispiel in der Automobilbranche und der Medizinbranche. An die Produktion werden in diesen Bereichen hohe Ansprüche im Bereich der Präzision gestellt. Um diese Versprechungen und Anforderungen seitens der Kunden einzuhalten und zu erfüllen nutzen international viele Unternehmen die Produkte von Blum, mit Hilfe derer die Präzision der Fertigung geprüft werden kann. Aus den Präzisionsansprüchen an die Kunden-Werkstücke resultieren folglich auch hohe Ansprüche an die Geräte und Software, die in der Qualitätssicherung verwendet werden. \\
Die Arbeit an einem Projekt/Produkt ist eigentlich nie wirklich vorbei. 
\\Das hat mehrere Gründe: Es werden immer wieder Fehler gefunden, die behoben werden müssen, um die Qualitätsversprechen und Erwartungen einzuhalten. Außerdem kommen immer wieder Anfragen von Kunden, die sofern es wirtschaftlich ist, umgesetzt werden sollten, um die Kunden nicht an andere Hersteller zu verlieren. Und auch den Entwicklern kommen immer wieder neue Ideen für neue Funktionen haben, die das Einsatzgebiet der Produkte erweitern.\\
Aus diesen Gründen wird die Konfiguration von Produkten immer aufwändiger und komplexer. Um dieser Problematik entgegen zu wirken wird ein "Parametertabellenkonfigurator" erstellt.  
\\Der Bereich der Produktpflege und -erweiterung wird durch die Mitarbeit am bestehenden Produkt "LC-Vision" thematisiert.
\\Beide Projekte dienen dazu den Einstieg in die Abläufe der Softwareentwicklung verständlich und praxisnah kennenzulernen.
\newpage   
\section{Softwareentwicklung}
Die Softwareentwicklung besteht nicht nur aus der reinen Programmierung von Software. Der Prozess fängt schon früher mit der Anforderungsanalyse an. Dabei ist es wichtig aus den Wünschen und Erwartungen die für einen Prototypen relevanten Informationen herauszufiltern. Nach dem man die wichtigsten Informationen bekommen hat, beginnt die Arbeit an einem Prototypen. Dieser kann, wenn ein "Graphical User Interfce"(GUI) vorgesehen ist, mit einem Mockup-Tool erstellt werden. Zusätzlich können erste Funktionen und  Schnittstellen konzipiert werden. Spielt ein GUI in der Planung keine Rolle, werden lediglich zentral Schnittstellen und Funktionen geplant. \\Dieser Grobentwurf kann in direkter Zusammenarbeit mit dem Kunden erfolgen, oder es finden in regelmäßigen Abständen Termine statt, bei denen die Gedanken und Entwürfe der Entwickler besprochen werden. Die zweite Option ist in der Praxis weitaus häufiger vertreten, da sie (in den meisten Fällen) weniger Zeit braucht und somit Kosteneffizienter ist. Das liegt unter anderem daran, dass weniger Zeit durch Diskussionen über Kleinigkeiten verloren geht.\\
Nach dem eine Einigung für einen Prototypen gefunden wurde, kann mit der eigentlichen Entwicklungsarbeit begonnen werden. Hier werden die Ideen vom Prototyp übernommen und weitere notwendige Funktionen entwickelt. Häufig fallen während der Entwicklung der Details Änderungen an, die man beim Prototyp nicht beachtet hat. In diesen Fällen muss man entscheiden, ob man sich mit dem Kunden in Verbindung setzt um die Änderungen zu besprechen, oder ob man die Änderung einfach vornimmt
\\Diese Entscheidung ist meistens davon abhängig, wie groß die nötige Änderung ist und davon, wie die Betreuung seitens des Kunden aussieht. Wenn es eine Ansprechperson gibt, die die nötige Zeit und das benötigte Wissen hat, so dass in kurzer Zeit mit einer hilfreichen Antwort gerechnet werden kann, spricht nichts gegen genaue Absprachen. Ist aber mindestens ein Faktor nicht gegeben, sollte man, wenn möglich, mit Projektbeteiligten innerhalb der eigenen Firma über die notwendigen Änderungen sprechen, oder es ansonsten selbstständig entscheiden. Denn das Warten auf eine hilfreiche Antwort der Ansprechperson, vor allem im späteren Verlauf, wenn ein großer Teil schon fertig ist, kann eine lange Verzögerung mit sich bringen.
\\Wenn dann die Entwicklungsarbeit fertig ist, geht es in die Evaluierung mit dem Kunden. Hier wird besprochen, ob die Anforderungen ausreichend gut umgesetzt wurden. 
\\Ist das der Fall, kann das Produkt für den Einsatz freigegeben werden. Die Freigabe ist der Start in die ebenso wichtige Phase der Produktpflege. Hier gilt es auftretende Fehler zu beheben, um die Funktionalität zu gewährleisten. Falls das bestehende Produkt nur als Grundbaustein konzipiert war, geht es in dieser Phase auch darum neue Funktionen zu implementieren. Diese Entwicklungsschritte laufen dann wieder nach dem beschriebenen Konzept ab, in dem man Anforderungen bespricht, Grobentwürfe erstellt, diese implementiert und dann prüft, ob die Anforderungen erfüllt wurden. 
\\Diese letzte Phase der Produktpflege und Weiterentwicklung dauert in den meisten Fällen so lange wie das Produkt wirtschaftlich rentabel ist. 
\section{Parametertabellenkonfigurator}
\subsection{Problemstellung}
Bei bestehenden Programmen für numerische Steuerungen müssen während der Inbetriebnahme Parametertabellen in einem beliebigen Texteditor angepasst werden. Aufgrund der Anzahl, wie auch der Einstellmöglichkeiten der Parameter, wird hierfür eine separate Installationsanleitung benötigt. Der Prozess der Parametrierung ist daher Fehleranfällig und von der Erfahrung des Inbetriebnehmers abhängig.Um den hohen Anforderungen der Kunden gerecht zu werden, gilt es potentielle Fehlerquellen zu eliminieren. Hierfür soll selbstständig ein "Parametertabellenkonfigurator“ entwickelt werden.\newpage
\subsection{Anforderungen}
 

\subsection{Entwurf Benutzeroberfläche}
Nach dem alle Anforderungen besprochen war, konnte mit den Überlegungen für die Benutzeroberfläche begonnen werden. \\
Um die Überlegungen schnell in einen sichtbaren Entwurf zu übertragen, wird das Programm "Balsamiq Mockups 3" verwendet. Mit diesem Tool können alle gängigen Designs nachgebaut werden und auch mit Funktionalität, zum Beispiel das anklicken von Buttons oder das auswählen von Optionen in Drop-down Menüs können somit schnell und einfach visualisiert werden. Das ist hilfreich, um Designvorschläge und Ideen besprechen zu können, ohne das man die Änderungen aufwendig programmieren muss.
\\Zusätzlich bietet der erstellte Entwurf eine gute Unterstützung für die Implementierung des Designs, da man sich den Aufbau der Software in Bezug auf Menüs und Buttons nicht merken muss, sondern diesen immer konkret vorliegen hat.
\\Für den "Parametertabellenkonfigurator" fiel die Wahl auf ein Design, dass aus mehreren Fenstern aufgebaut ist, um die Bedienbarkeit möglichst einfach zu halten, denn die Anzahl an Auswahlmöglichkeiten würde innerhalb eines Fensters die Übersichtlichkeit deutlich reduzieren.
\\ Wäre alles in einem Fenster, müsste durch "ausgrauen" oder ausblenden der Optionen dargestellt werden, welche momentan verfügbar sind. Das ist zwar möglich, ersteres führt aber bei manchen Anwendern zu Verwirrung, oder Unsicherheiten, was den Konfigurationsprozess der Produkte von Blum negativ beeinflussen könnte, zweites ist optisch nicht ansprechend, da immer wieder unerwartete, freie Flächen entstehen und eventuell Layouts verschoben werden. \\
Diese Problematik kann durch das aufteilen in mehrere Fenster behoben werden, da nur die im Fenster angezeigten Optionen auch verfügbar sind. 
\\Da von Anfang mindestens ein zweites Fenster geplant war, um die Beschreibungen der einzelnen Parameter in einer vergrößerten Ansicht anzeigen zu können, fügte sich auch das Fenster für die Auswahl der benötigten Parametertabelle problemlos in das Design ein. 
Das Hauptfenster ist in zwei Bereiche unterteilt. Der obere ist für die Parametertabelle allgemein, er besteht aus eine Anzeige, um zu sehen welche Parametertabelle ausgewählt ist, ein Button um eine andere Parametertabelle auswählen zu können und ein Feld im der der Inhalt angezeigt wird und bearbeitet werden kann. Um den Einrichtungsprozess zu unterstützen, werden alle bearbeiteten Zeilen farblich markiert. 
\\Im unteren Teil geht es um die einzelnen Parameter. Es gibt eine Anzeige für den aktuell ausgewählten Parameter, einen Bereich um die Beschreibung, wenn vorhanden mit Bild, anzuzeigen und einen Button um die Beschreibung isoliert und vergrößert in einem neuen Fenster zu öffnen. Um auch hier die Bedienbarkeit zu erleichtern gibt es zwei weitere Buttons die es ermöglichen schnelle zu vorherigen, beziehungsweise dem nachfolgenden Parameter zu springen.
%TODO einfügen von Bildern
\subsection{Programmierung}
\subsubsection{Software-Struktur}
\subsubsection{Verarbeiten von Benutzeraktionen}
\subsubsection{Herausforderungen und Lösungen}
\subsection{Tests}
\subsection{geplante Erweiterungen}
\subsection{Reflexion}

\section{LC-Vision}


\end{document}