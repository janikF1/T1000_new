\documentclass[10pt,a4paper]{article}
\usepackage[utf8]{inputenc}
\usepackage{amsmath}
\usepackage{amsfonts}
\usepackage{amssymb}
\author{Janik Frick}
\title{T1000}
\begin{document}
%TODO Einstellungen machen
%TODO Deckblatt entwerfen
%TODO Eigenarbeitserklärung
%TODO Abkürzungsverzeichnis
%TODO Logos auf jede Seite

\tableofcontents
\newpage
\section{Einleitung}
Die Blum-Novotest GmbH(Blum) ist mit ihren Produkten aus dem Bereich der Mess- und Prüftechnik in anspruchsvollen Industrien vertreten, dazu gehören unter anderem die Automobilindustrie und die Gesundheitsindustrie. \\
Die hohen Ansprüche an Qualität und Funktionalität der Produkte der Endkunden heben die Ansprüche der Produzenten an die Produkte von Blum ebenfalls auf ein hohes Niveau, denn die Produkte von Blum werden eingesetzt, um die Qualitätsstandards zu kontrollieren und zu halten. \\

\section{Parametertabellenkonfigurator}
In sogenannten Parametertabellen(Paratabs) befinden sich die Zyklen und Informationen, mit denen die Messwerkzeuge konfiguriert und gesteuert werden. 
\subsection{Problemstellung}
Bei bestehenden Programmen für numerische Steuerungen müssen derzeit während der Inbetriebnahme Parametertabellen in einem beliebigen Texteditor angepasst werden. Aufgrund der Anzahl, wie auch der Einstellmöglichkeiten der Parameter, wird hierfür eine separate Installationsanleitung benötigt. Der Prozess der Parametrierung ist daher Fehleranfällig und von der Erfahrung des Inbetriebnehmers abhängig.Um den hohen Anforderungen der Kunden gerecht zu werden, gilt es potentielle Fehlerquellen zu eliminieren. Hierfür soll selbstständig ein „Parametertabellenkonfigurator“ entwickelt werden.
\subsection{Anforderungsanalyse}
\subsubsection{Abgrenzung Minimum Viable Product}
\subsection{verwendete Technologien}
Positio anpassen
\subsection{Vorgehen}
\subsection{Entwurf Benutzeroberfläche}
\subsection{Programmierung}
\subsubsection{Software-Struktur}
\subsubsection{Verarbeiten von Benutzeraktionen}
\subsubsection{Herausforderungen und Lösungen}
\subsection{Tests}
\subsection{geplante Erweiterungen}
\subsection{Reflexion}

\section{LC-Vision}


\end{document}